%This is where the body of your abstract goes, limited to 150 words
%for a thesis, and 350 words for a dissertation or document.  The
%word count limits apply to the regular Abstract in the thesis and
%to the separate Special Abstract.  Use the same text for both; just
%adjust the margins and heading.  The abstract should summarize your
%work.  The UMI booklet listed in the resources section of the U of
%A manual provides some writing tips.  The abstract for a dissertation
%or document may be longer than one page; word count is more important
%than page length in this section.

%If you are doing a paper submission, submit one copy of the special 
%abstract, and two extra copies of your title page, in the box with 
%the final copies of your thesis.  If you are doing an electronic
%submission, you can ignore the special abstract.

The goal of this thesis is to explore the rich subject of magnetism, spin and dipole moments in physics from both a classical and quantum mechanical perspective. We explore how models of magnetic dipoles can be expanded upon or improved and apply our ideas to physical systems at both the small scale of atomic systems to cosmic scales involving the universe as a whole.

In classical physics, we propose covariant models of magnetic dipoles which as well as extensions to classical torque equations when particles with anomalous magnetic moments are considered. The theoretical implications of Gilbert versus Amperian-like dipoles are considered as well as their linked relationship in relativistic mechanics.

Quantum mechanically, we compare and contrast the Dirac-Pauli (DP) and Klein-Gordon-Pauli (KGP) methods of introducing anomalous dipole moments into relativistic quantum mechanics. We cover how the two approaches are incompatible as well as provide arguments for why we prefer the KGP approach in our study of dipole moments. This comparison is made directly using the situation of homogeneous magnetic fields as hydrogen-like systems.

Taking inspiration from studying charged systems, we study models of magnetic moment for neutral particles such as the neutrino. We take special interest in the potential interplay of CP violation and magnetic dipole phenomenon by exploring modifications to the Jarlskog invariant which characterizes CP violation from flavor states.

The final chapter tackles cosmic scale magnetism in the universe which is not fully understood and is intensely researched today. We explore the possibility of spin polarization of the early universe primordial gas as a mechanism for cosmic magnetogenesis. We forward the idea that the electron-positron epoch of the early universe is deserving of further attention and may be connected to the residue fields we see today.


