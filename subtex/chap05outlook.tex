%%%%%%%%%%%%%%%%%%%%%%%%%%%%%%%%%%%%%%%
\chapter{Outlook}
\label{chap:outlook}
%%%%%%%%%%%%%%%%%%%%%%%%%%%%%%%%%%%%%%%

%%%%%%%%%%%%%%%%%%%%%%%%%%%%%%%%%%%%%%%
\section{Key results}
\label{sec:results}
%%%%%%%%%%%%%%%%%%%%%%%%%%%%%%%%%%%%%%%
{\xred Much to do! Workshop thoughts. Make more specific. Look at final arXiv preprint for specificity.}

We characterized the primordial magnetic properties of the early universe before recombination. We studied the temperature range of $2000\keV$ to $20\keV$ where all of space was filled with a hot dense electron-positron plasma (to the tune of 450 million pairs per baryon) which occurred within the first few minutes after the Big Bang. We note that our chosen period also includes the era of Big Bang Nucleosynthesis.

We found that subject to a primordial magnetic field, the early universe electron-positron plasma has a significant paramagnetic response due to magnetic moment polarization. We considered the interplay of charge chemical potential, baryon asymmetry, anomalous magnetic moment, and magnetic dipole polarization on the nearly homogeneous medium. We find that electron-positron magnetization rapidly vanishes as the number of pairs depletes as the universe cools. This therefore presents an opportunity for induced currents to facilitate inhomogeneities in the early universe. We also presented a simple model of self-magnetization of the primordial electron-positron plasma which indicates that only a small polarization asymmetry is required to generate significant magnetic flux when the universe was very hot and dense.
%%%%%%%%%%%%%%%%%%%%%%%%%%%%%%%%%%%%%%%
\section{Future research efforts}
\label{sec:future}
%%%%%%%%%%%%%%%%%%%%%%%%%%%%%%%%%%%%%%%
