Before we handle an ensemble system, we will look at the magnetic moment $\boldsymbol{\mu}$ of a single-particle quantum system. To avoid confusion, the permeability will always be denoted either by a subscript for the vacuum or medium to always differentiate from magnetic moment. If we apply an external field with the local constant value of $B$ which is primarily responsible for the particle's response, the magnetic moment can be evaluated from the matrix element
\begin{alignat}{1}
  \label{CHIeq03} \left|\boldsymbol{\mu}_{n}\right|=-\left\langle n\left|\partial\hat{\mathcal{H}}/\partial B\right|n\right\rangle=-\frac{\partial E_{n}}{\partial B}\,,
\end{alignat}
where $\hat{\mathcal{H}}$ is the Hamiltonian of the system. It is valuable to point out that the individual dipole's response within a medium is only uniquely dependent on the external field $\textbf{H}$ in the case of weak medium magnetization. However, if the bulk magnetization is may easily be large and thus influential to each individual dipole. Physically each dipole is sensitive to the total magnetic flux $\textbf{B}$ which includes a mixture of external field and bulk magnetization $\textbf{M}$ from its neighbors. The magnetization density of the quantum system is then
\begin{alignat}{1}
  \label{CHIeq04} M_{n}(B)=\frac{1}{V}\left|\boldsymbol{\mu}_{n}\right|\,.
\end{alignat}
If we couple the system to a thermal reservoir of temperature $T$, the averaged magnetization at thermal and chemical equilibrium is
\begin{alignat}{1}
  \label{CHIeq05} M(B,T,\eta)=\frac{\sum_{n}M_{n}e^{-\beta (E_{n}-\eta N)}}{\sum_{n}e^{-\beta (E_{n}-\eta N)}}\,,
\end{alignat}
where $\beta=1/k_{B}T$, $k_{B}$ is the Boltzmann constant and $\eta$ is the chemical potential. We then introduce the grand potential $\Phi$ defined by
\begin{alignat}{1}
  \label{CHIeq06} \Phi=-\frac{1}{\beta}\ln\left({\sum_{n}e^{-\beta (E_{n}-\eta N)}}\right)=-\frac{1}{\beta}\ln\left(\mathcal{Z}\right)\,,
\end{alignat}
where $\mathcal{Z}$ is the grand partition function. This allows us to rewrite eq.~\eqref{CHIeq05} as
\begin{alignat}{1}
  \label{CHIeq07} M(B,T,\eta)=-\frac{1}{V}\frac{\partial \Phi}{\partial B}\,.
\end{alignat}
Combining eq.~\eqref{CHIeq06} and eq.~\eqref{CHIeq07} in the grand ensemble yields
\begin{alignat}{1}
  \label{Mag} M=\frac{1}{\beta V}\frac{\partial}{\partial B}\ln\left(\mathcal{Z}\right)\,.
\end{alignat}
The magnetic susceptibility is related to the magnetization via
\begin{alignat}{1}
  \label{CHIeq09} \chi=\mu_{vac.}\frac{\partial M}{\partial B}\,.
\end{alignat}
If a given thermodynamic system is well described by a partition function, we can evaluate the susceptibility using eq.~\eqref{CHIeq09}.








We note here one important difference between KGP and DP eigen-energies in the context of cosmology: The anomalous magnetic moment portion of the DP statistics is suppressed by $1/a(t)$ over cosmological time while the AMM contribution is preserved in the KGP model. That the universe's expansion makes a distinction between $g=2$ magnetic moment and AMM for DP fermions appears as a rather artificial and nonphysical trait. While the suppression of AMM may often be small for particles such as electrons, this suppression is non-trivial for particles with large AMM values such as the proton. That cosmological redshift would push DP protons to be described by $g=2$ eign-energies in the non-relativistic limit counts as a malaise for the model and further strengthens our thinking that the KGP model is more appropriate for cosmological studies. Motivated by \req{XScale}, we can introduce a dimensionless cosmic magnetic scale which is frozen in the homogeneous case as
\begin{alignat}{1}
    \label{Bo} b_{0}\equiv\frac{eB}{T^{2}}=\frac{eB\hbar c^{2}}{(k_{B}T)^{2}}\ \mathrm{(S.I)}\,,
\end{alignat}
where we've included the expression explicitly in full SI units. We can estimate the value of $b_{0}$ from the bounds on the extra-galactic magnetic field strength and the temperature of the universe today.  If the origin of deep space extra-galactic magnetic fields are relic fields from the early universe, which today are expected to exist between $5\times10^{-12}\ \mathrm{T}>B_{relic}>10^{-20}\ \mathrm{T}$, then at temperature $T=2.7\ \mathrm{K}$, the value of the cosmic magnetic scale is between
\begin{alignat}{1}
    \label{BoScale} 5.5\times10^{-5}>b_{0}>1.1\times10^{-11}\,.
\end{alignat}
This should remain constant in the universe at-large up to the last epoch the universe was sufficiently magnetized to disturb this value. As the electron-proton plasma which generated the CMB was relatively dilute over its duration, it was unlikely sufficiently magnetized to significantly alter this value over extra-galactic scales. Rather, the best candidate plasma to have been sufficiently magnetized and dense to have set the relic field magnetic scale would have been the electron-positron plasma which existed during the duration of Big Bang Nucleosynthesis (BBN) and beforehand.

As $b_0$ is a constant of expansion, assuming the electron-proton plasma between the CMB and electron-positron annihilation did not greatly disturbed it, we can calculate the remnant values at the temperature $T=50\ \mathrm{keV}$ with the expression
\begin{align}
  \label{BBNFields} B(T)=\frac{b_{0}}{e}T^{2}\,,
\end{align}
yielding a range of field strengths
\begin{align}
  \label{BBNRange} 2.3\times10^{5}\ \mathrm{T}>B(T=50\ \mathrm{keV})>4.6\times10^{-4}\ \mathrm{T}\,,
\end{align}
during which the electron-positron plasma in the universe had a number density comparable to that of the Solar core with $n_{e}=4.49\times10^{24}\ \mathrm{cm}^{-3}$ at $r=0.01R_{\odot}$.

While we consider the $g$-factor to be the immutable description of magnetic moment, it is useful in the case of Landau levels to consider the anomaly $a$ as from the energies given in eq.~\eqref{LANeq02}, $g$ always is linearly added or subtracted from the Landau quantum number which are integers.

Degeneracy is restored for values of the anomalous parameter given in eq.~\eqref{LANeq04}. While generating a large number of eccentric states from large anomalous moment may be of theoretical interest, it is of practical interest to consider particles like the proton or electron where only the ground state is uniquely disturbed. Due to the properties of logs, the overall partition function will be sum of the partition function of each species, which allows us to consider each species separately.

The magnetization of this term, defined by \req{Mag}, is found to be
\begin{alignat}{1}
    \label{FreelikeMag} M_{F}^{s,\sigma}=\frac{T}{V}\frac{\partial m_{s}}{\partial B}\frac{\partial}{\partial m_{s}}\ln(\mathcal{Z}_{F}^{\sigma})\,\,,
\end{alignat}
with the total magnetization given by the sum over the four species
\begin{alignat}{1}
    \label{TotalFreeMag} M_{F}=\sum_{s,\sigma}M_{F}^{s,\sigma}\,.
\end{alignat}
The term by term magnetization evaluates as
\begin{multline}
  \label{MagExplicit} M_{F}^{s,\sigma} = \frac{e(1-gs)}{2m_{s}}\frac{5}{m_{s}}\ln\left(\mathcal{Z}^{\sigma}_{F}\right)|_{m_{s}}\\
  -\frac{e(1-gs)}{2m_{s}}\frac{1}{3}\frac{L^{3}}{(2\pi)^{2}}m_{s}^{3}\left(\frac{m_{s}}{T}\right)^{2}
  \int_{-\infty}^{+\infty}dt\left[z_{\sigma}^{-1}e^{a_{s}\cosh(t)}\cosh(t)\sinh^{4}(t)F^{2}\left[X_{s},\sigma\right]\right]\,,
\end{multline}