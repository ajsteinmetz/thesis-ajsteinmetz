\chapter{Classical magnetic dipole moments}
    \section{Stern-Gerlach force}
        \subsection{Amperian and Gilbert dipoles}
    \section{TBMT equations}
    \section{Magnetic spin potential}
%\\\\\\\\\\\\\\\\\\\\\\\\\\\\\\\\\\\\\\\\\\\\\\\\\\\\\\\\\\\\\\\\\\\\\\\\\\\\\\\\\\\\\\\\\\\\\\\\\\\\\\\\\\\\\\\\\\\\\\\\\\\\\\\\\\\\\\\\\\\\\\%
In classical theory, we consider the covariant dipole force which acts upon a particle due to its intrinsic magnetic moment. The generalized covariant Lorentz and dipole force is given by
\begin{alignat}{1}
  \label{LSG01} m\frac{\mathrm{d}u^{\alpha}}{\mathrm{d}\tau}&=eF^{\alpha\beta}u_{\beta}+dG^{\alpha\beta}u_{\beta}\,,\\
  \label{LSG02} G^{\alpha\beta}&=\partial^{\alpha}F^{*\beta\gamma}s_{\gamma}-\partial^{\beta}F^{*\alpha\gamma}s_{\gamma}\,,
\end{alignat}
where $e$ and $d$ are the electric and dipole charges, and $F^{*}$ is the electromagnetic dual tensor. While the first term in equation~\eqref{LSG01} is the standard Lorentz force, the second term is a covariant formulation of the Stern-Gerlach (SG) force. Because the spin precession is sensitive to the force on a particle, the presence of a SG force will induce precession terms which are second order in spin. The generalized spin precession that corresponds to eq.~\eqref{LSG01} is therefore
\begin{alignat}{1}
  \notag\frac{\mathrm{d}s^{\mu}}{\mathrm{d}\tau}&=(1+\tilde{a})\frac{e}{m}F^{\mu\nu}s_{\nu}-\tilde{a}\frac{e}{m}u^{\mu}\left(u_{\alpha}F^{\alpha\beta}s_{\beta}\right)/c^{2}\\
  \label{LSG03}&+(1+\tilde{b})\frac{d}{m}G^{\mu\nu}s_{\nu}-\tilde{b}\frac{d}{m}u^{\mu}\left(u_{\alpha}G^{\alpha\beta}s_{\beta}\right)/c^{2}\,.
\end{alignat}
The constants $\tilde{a}$ and $\tilde{b}$ are arbitrary allowing for extra terms not forbidden by special relativity. In the standard derivation of relativistic spin precession, in the form of the TBMT equation, the $\tilde{a}$ constant is associated with the anomalous magnetic moment. In allowing for spin precession sourced by a Stern-Gerlach dipole force, an additional constant $\tilde{b}$ must be introduced. The terms in eq.~\eqref{LSG02} involving the $G$ tensor are spin precession directly originating from dipole forces. In homogeneous electromagnetic fields, eq.~\eqref{LSG02} reduces to the standard TBMT equation.

        \subsection{Modified TBMT equations}
        \subsection{Unified Amperian and Gilbert dipoles}
        \subsection{Dynamic particle motion}
            \subsubsection{Charged particles}
            \subsubsection{Neutral particles}